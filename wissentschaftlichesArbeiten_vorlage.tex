\documentclass[12pt]{article}

%Allgemeine Einstellungen

%Abstände
\usepackage[a4paper,left=3cm,right=3cm,top=3cm,bottom=3.5cm,headsep=12pt]{geometry}%Bottom extra 0.5cm für Footer

%Deutsches Sprachpacket
\usepackage[german,ngerman]{babel}

%Times New Roman
\usepackage{mathptmx}

%Titelseite einbinden
\usepackage{pdfpages}

%1.5-Zeilenabstand
\usepackage[onehalfspacing]{setspace}

%Stil der Überschriften, siehe ueberschriften.sty
\usepackage[numeric]{ueberschriften}

%Stil des Inhaltsverzeichnisses, siehe inhaltsverzeichnis.sty
\usepackage[numeric]{inhaltsverzeichnis}

%Abkürzungsverzeichnis, siehe abk_verzeichnis.sty
\usepackage{abk_verzeichnis}

%Stil der Fußzeilen, siehe fusszeilen.sty
\usepackage{fusszeilen}

%Literaturverzeichnis und Zitate, siehe literatur.sty
\usepackage{literatur}

%Stil für Header und Footer, siehe header_footer.sty
%Wenn nicht erwünscht, müssen auch die Befehle \frontmatter, \mainmatter auskommentiert werden
\usepackage{header_footer}

%Stile für Code-Ausschnitte, siehe codes.sty
\usepackage{codes}

%Stile für Anhänge, Bilder, ...
\usepackage{anhang}

%Silbentrennung (manche Worte werden am Zeilenende nicht getrennt, diese müssen dann nachgetragen werden)
\usepackage[T1]{fontenc}
\hyphenation{öf-fent-lich-en}

%DEBUGGING (Zeigt Boxen an)
%\usepackage{showframe}

\begin{document}

\renewcommand{\mytitle}{Governanceethik und\\moralische Anreize}%Titel für oben links
\renewcommand{\myauthor}{Lennart Schulte-Kellinghaus,\\Timo Stovermann}%Name für unten links
\renewcommand{\headheight}{27pt}%Bei Mehrzeiligem Titel muss Headerhöhe angepasst werden

\includepdf[pages={1-}]{titelseite.pdf}

\frontmatter%Stil des Headers/Footers ändern

\pagenumbering{Roman}

\addcontentsline{toc}{part}{Abkürzungsverzeichnis}%Abk-Verz. ins Inhaltsverzeichnis
\printabbreviations%abk_verzeichnis.sty
\clearpage
\renewcommand{\plaintitle}{Abbildungsverzeichnis}
\addcontentsline{toc}{part}{Abbildungsverzeichnis}
{\def\makebox[#1][#2]#3{#3}%
\listoffigures
}
\clearpage
\renewcommand{\plaintitle}{Tabellenverzeichnis}
\addcontentsline{toc}{part}{Tabellenverzeichnis}
{\def\makebox[#1][#2]#3{#3}%
\listoftables
}
\clearpage
\renewcommand{\plaintitle}{Inhaltsverzeichnis}%Titel für oben Rechts
%Defbox, damit gepunktete Linie bis zur Zahl geht
{\def\makebox[#1][#2]#3{#3}%
	\tableofcontents
}

\addtocontents{toc}{\vspace{24pt}}%Freiraum im ToC

\clearpage
\mainmatter%Stil des Headers/Footers ändern
\pagenumbering{arabic}

\part{Theoretische Grundlagen}
Ethik spielt in Unternehmen eine immer größer werdende Rolle. Im Kontrast zum 20. Jahrhundert, als beispielsweise hohe Emissionswerte eines Unternehmens kaum Beachtung gefunden haben, gehen die CO2-Emissionen von Unternehmen seit Jahren stetig zurück. Genauso hat die Gleichberechtigung von Männern und Frauen immer mehr Präsenz in der Wirtschaft gewonnen, in form von Anstrengungen den Gender-Pay-Gap zu eliminieren oder den Frauenanteil in Management-Positionen zu erhöhen. Diese beiden Veränderungen sind neben vielen anderen maßgeblich durch Ethik bestimmt, da sie keinen (direkten) wirtschaftlichen Nutzen schaffen. In wie fern die Ethik und gesellschaftliche wie moralische Aspekte die Entscheidungen eines Unternehmens beeinflusst, soll hier kurz erörtert werden.
\section{Governance}
Gegenüber der klassischen Unternehmensführung versteht man unter dem Begriff Governance nicht-hierarchische Formen der Steuerung, in denen eine Verknüpfung verschiedener Ebenen und Perspektiven im Fokus steht. Im wirtschaftlichen Kontext wird der Begriff Corporate Governance auf Prinzipien innerhalb eines Unternehmens angewendet, welche den rechtlichen und faktischen Ordnungsrahmen für die Leitung und Überwachung festlegen. Der Zweck dieser Prinzipien ist es, Interessenkonflikte zwischen Shareholdern und Stakeholdern zu schlichten und das Unternehmen für alle Akteure ansprechend zu gestalten, um opportunistisches Verhalten einzuschränken. Dabei spielen neben ökonomischen auch moralische Anreize eine Rolle.
\section{Systemtheoretische Zusammenhänge}
Die wirtschaftliche wie soziale Gesellschaft besteht aus vielen Teilbereichen, welche teilweise unterschiedliche Ansichten zu den gleichen sozio-ökonomischen Fragestellungen vertreten. Ein solches Funktionssystem hat jeweils einen “Leitcode”, eine Sprache, über welche es gesteuert wird. Bei der Wirtschaft ist dies der Angebot-Nachfrage-Mechanismus mit dem Kommunikationsmittel Geld. Ein Funktionssystem kann nur über seinen Leitcode kommunizieren, was das Zusammenwirken von mehreren Funktionssystemen zur selben Problematik zunächst unmöglich macht. Daraus folgt, dass die Moral als Zweck an sich im reinen Wirtschaftssystem keinen Platz hat, da sie an sich keine monetäre Wertschöpfung bewirkt. Deshalb wird neben den Funktionssystemen das Konstrukt der Organisationssysteme eingeführt. Diese sind in der Lage, mit mehreren Funktionssystemen zu kommunizieren und Diskurse zwischen diesen zu moderieren.
\section{Die Anreiz-Dimensionen}
Anreize können anhand zweier Dimensionen kategorisiert werden, der Art der primären Hintergründe und die Herkunft des Anreizes. Die Hintergründe können ökonomisch, also primär auf das Eigeninteresse des Unternehmens bezogen, sein oder moralisch, welche “nicht nur aufgrund externer Belohnung [...] befolgt” werden sondern auf gesellschaftlichen Normen basieren. Auf der zweiten Dimension können externe Akteure (Kunden, Partner, Staat, …) oder unternehmensinterne Beweggründe einen Anreiz hervorrufen bzw. belohnen. Daraus ergibt sich eine Matrix mit 4 Anreiz-Kategorien.
\begin{center}
\begin{tabular}{|p{3cm}|p{5cm}|p{5cm}|}
\hline
Anreize & \textbf{extrinsisch} & \textbf{intrinsisch}\\\hline
\textbf{ökonomisch} & extrinsisch-ökonomisch & intrinsisch-ökonomisch\\\hline
\textbf{moralisch} & extrinsisch-moralisch & intrinsisch-moralisch\\\hline
\end{tabular}
\end{center}
\part{Treiber moralischen Handelns}
\clearpage
\frontmatter%Stil des Headers/Footers ändern
\renewcommand{\plaintitle}{Literaturverzeichnis}
\pagenumbering{Roman}
\setcounter{page}{5}
\addtocontents{toc}{\vspace{24pt}}
\addcontentsline{toc}{part}{Literaturverzeichnis}%Literatur-Verz. ins Inhaltsverzeichnis
\printMyBibliography

\end{document}