\documentclass[12pt]{article}

%Allgemeine Einstellungen

%Abstände
\usepackage[a4paper,left=3cm,right=3cm,top=3cm,bottom=3.5cm,headsep=12pt]{geometry}%Bottom extra 0.5cm für Footer

%Deutsches Sprachpacket
\usepackage[german,ngerman]{babel}

%Times New Roman
\usepackage{mathptmx}

%Titelseite einbinden
\usepackage{pdfpages}

%1.5-Zeilenabstand
\usepackage[onehalfspacing]{setspace}

%Stil der Überschriften, siehe ueberschriften.sty
\usepackage[numeric]{ueberschriften}

%Stil des Inhaltsverzeichnisses, siehe inhaltsverzeichnis.sty
\usepackage[numeric]{inhaltsverzeichnis}

%Abkürzungsverzeichnis, siehe abk_verzeichnis.sty
\usepackage{abk_verzeichnis}

%Stil der Fußzeilen, siehe fusszeilen.sty
\usepackage{fusszeilen}

%Literaturverzeichnis und Zitate, siehe literatur.sty
\usepackage{literatur}

%Stil für Header und Footer, siehe header_footer.sty
%Wenn nicht erwünscht, müssen auch die Befehle \frontmatter, \mainmatter auskommentiert werden
\usepackage{header_footer}

%Stile für Code-Ausschnitte, siehe codes.sty
\usepackage{codes}

%Stile für Anhänge, Bilder, ...
\usepackage{anhang}

%Silbentrennung (manche Worte werden am Zeilenende nicht getrennt, diese müssen dann nachgetragen werden)
\usepackage[T1]{fontenc}
\hyphenation{öf-fent-lich-en}

\hyphenation{apl-pha--num-me-rischen}

%DEBUGGING (Zeigt Boxen an)
%\usepackage{showframe}
\setlength{\skip\footins}{12pt}

%Für Zeilenumbrüche in Tabellenzellen
\usepackage{makecell}
%Falls Tabellen oder Bilder komisch in den Text fließen
\usepackage{placeins}

%Ab hier kommt spezifisches für dieses Dokument
\definecolor{grayBack}{rgb}{0.8,0.8,0.8}

\usepackage{verbatim}

\begin{document}

\renewcommand{\mytitle}{Dokumentation/Tutorial}%Titel für oben links
\renewcommand{\myauthor}{Dr. Frank N. Furter}%Name für unten links

\includepdf[pages={1-}]{titelseiteAlternative.pdf}

\frontmatter%Stil des Headers/Footers ändern

\pagenumbering{Roman}

\addcontentsline{toc}{part}{Abkürzungsverzeichnis}%Abk-Verz. ins Inhaltsverzeichnis
\printabbreviations%abk_verzeichnis.sty
\clearpage
\renewcommand{\plaintitle}{Abbildungsverzeichnis}
\addcontentsline{toc}{part}{Abbildungsverzeichnis}
{\def\makebox[#1][#2]#3{#3}%
\listoffigures
}
\clearpage
\renewcommand{\plaintitle}{Tabellenverzeichnis}
\addcontentsline{toc}{part}{Tabellenverzeichnis}
{\def\makebox[#1][#2]#3{#3}%
\listoftables
}
\clearpage
\renewcommand{\plaintitle}{Inhaltsverzeichnis}%Titel für oben Rechts
%Defbox, damit gepunktete Linie bis zur Zahl geht
{\def\makebox[#1][#2]#3{#3}%
	\tableofcontents
}

\addtocontents{toc}{\vspace{24pt}}%Freiraum im ToC

\clearpage
\mainmatter%Stil des Headers/Footers ändern
\pagenumbering{arabic}


\part{Wie kann ich dieses Template benutzen?}

\section{Überschriften und Inhaltsverzeichnis}

\subsection{Wie kann ich Überschriften auf verschiedenen Ebenen erzeugen?}
Es können Überschriften auf 6 Ebenen erzeugt werden:
\begin{verbatim}
\part{Ebene 1}
\section{Ebene 2}
\subsection{Ebene 3}
\subsubsection{Ebene 4}
\paragraph{Ebene 5}
\subparagraph{Ebene 6}
\end{verbatim}

\noindent Wenn die Überschrift nicht nummeriert werden soll, muss der jeweilige Befehl mit einem * aufgerufen werden:
\begin{verbatim}
\section*{Ebene 1}
\end{verbatim}

\subsection{Wie kann ich die Nummerierung der Überschriften ändern?}
Die Überschriften können entweder nach einer rein nummerischen oder einer alphanummerischen Form nummeriert werden:
\begin{figure}[ht]
    \centering
    \begin{minipage}[t]{0.49\linewidth}
        \centering
        \includegraphics[width=\linewidth]{dokuImages/toc_num.png}
        \caption{numerisch}
    \end{minipage}% <- sonst wird hier ein Leerzeichen eingefügt
    \hfill
    \begin{minipage}[t]{0.49\linewidth}
        \centering
        \includegraphics[width=\linewidth]{dokuImages/toc_alphanum.png}
        \caption{alpha-numerisch}
    \end{minipage}
\end{figure}
Der Wechsel erfolgt in der \textit{.tex}-Datei im Preämbel:\\[6pt]
Für numerisch:
\begin{verbatim}
\usepackage[numeric]{ueberschriften}
\usepackage[numeric]{inhaltsverzeichnis}
\end{verbatim}
Für alpha-numerisch:
\begin{verbatim}
\usepackage[latour]{ueberschriften}
\usepackage[latour]{inhaltsverzeichnis}
\end{verbatim}

\subsection{Wie kann ich einen Eintrag zum Inhaltsverzeichnis hinzufügen?}
\begin{verbatim}
%Abk-Verz. ins Inhaltsverzeichnis
\addcontentsline{toc}{part}{Abkürzungsverzeichnis}
\end{verbatim}
Die Parameter sind das Element, zu dem hinzugefügt werde soll, die Ebene und der Titel.

\subsection{Wie kann ich das Inhaltsverzeichnis ausgeben?}
Um die gepunkteten Linien wirklich ganz bis zur Seitenzahl durchzuziehen, muss der Standard-Befehl noch in eine Box gewrappt werden:
\begin{verbatim}
{\def\makebox[#1][#2]#3{#3}%
	\tableofcontents
}
\end{verbatim}

\section{Abkürzungsverzeichnis}
Die Einträge im Abkürzungsverzeichnis werden automatisch aus der Datei \textit{abkuerzungen.csv} generiert. Dort sind sie in folgendem Format aufgeführt:
\begin{align*}
\text{abk}&;\text{bed};\\
\text{z.B.} &;\text{zum Beispiel};
\end{align*}
\noindent Um das Abkürzungsverzeichnis auf einer eigenen Seite auszugeben und auch im Inhaltsverzeichnis einzufügen:
\begin{verbatim}
\addcontentsline{toc}{part}{Abkürzungsverzeichnis}
\printabbreviations%abk_verzeichnis.sty
\clearpage
\end{verbatim}

\section{Bilder und Tabellen}
\subsection{Wie kann ich ein Bild im Text einbinden?}
\begin{verbatim}
\includegraphics[width=.9\textwidth]{./relativer/bildpfad.png}
\end{verbatim}

\subsection{Wie kann ich ein Bild im Abbildungsverzeichnis anzeigen?}
Um ein Bild im Abbildungsverzeichnis anzuzeigen, muss es in eine \textit{figure} gewrappt und mit einer Caption versehen werden:
\begin{verbatim}
\begin{figure}[ht]
    \centering
    \includegraphics[width=\linewidth]{relativer/bildpfad.png}
    \caption{Bild}
\end{figure}
\end{verbatim}
\noindent Die Option \textit{ht} gibt an, dass die Figure an diesem Ort stehen soll und nicht im Text fließen soll.

\section{Kopf- und Fußzeile}
Dieses Template bietet die Möglichkeit, folgende Informationen in Kopf- und Fußzeile anzuzeigen:
\begin{itemize}
\item oben-links: Titel der Arbeit
\item oben-rechts: aktueller Part (Überschrift Ebene 1)
\item unten-links: Autor
\item unten-rechts: Seite
\end{itemize}

\subsection{Wie kann ich den Titel und den Autor in der Kopf-/Fußzeile ändern?}
\begin{verbatim}
\renewcommand{\mytitle}{Dokumentation/Tutorial}%Titel für oben links
\renewcommand{\myauthor}{Dr. Frank N. Furter}%Name für unten links
\end{verbatim}
Bei einem Titel, der besser in zwei Zeilen passt, muss die Höhe der Kopfzeile angepasst werden:
\begin{verbatim}
\renewcommand{\mytitle}{Titel meiner Arbeit\\mit zwei Zeilen}
\renewcommand{\headheight}{27pt}
\end{verbatim}

\subsection{Wie kann ich den oben-rechts angezeigten Abschnitts-Titel ändern?}
Der Stil für die Header/Footer des Inhalts- und Literaturverzeichnisses sollen nicht auf den aktuellen Abschnittstitel zurückgreifen, sondern es soll der jeweilige Verzeichnisname angezeigt werden. Dieser wird in dem Befehl \textit{\textbackslash plaintitle} gespeichert/überschrieben. Die Art der Nummerierung kann ebenfalls gewechselt werden.
\begin{verbatim}
...

\frontmatter%Stil des Headers/Footers ändern

\pagenumbering{Roman}
...

\clearpage
\renewcommand{\plaintitle}{Abbildungsverzeichnis}
...

\clearpage
\renewcommand{\plaintitle}{Tabellenverzeichnis}
...

\clearpage
\renewcommand{\plaintitle}{Inhaltsverzeichnis}%Titel für oben Rechts
...

\clearpage
\mainmatter%Stil des Headers/Footers ändern

\part{Inhalt}
...

\clearpage
\frontmatter%Stil des Headers/Footers ändern
\renewcommand{\plaintitle}{Literaturverzeichnis}
\pagenumbering{Roman}
\setcounter{page}{5}
...
\end{verbatim}

\section{Literaturverzeichnis und Zitate}
LaTeX bietet zwar ein eigenes System zur Anzeige von Literatur an, darin sind Anpassungen aber sehr komplex/nicht so einfach umsetzbar. Deshab wurde für dieses Template ein eigenes System zur Literatur-Verwaltung entwickelt. Um dieses effektiv nutzen zu können, muss das Programm \textit{WA\_ LaTeX.exe} ausgeführt werden. Dies startet einen Webserver, welcher unter \textit{http://localhost:8081/overview} zu erreichen ist.

\subsection{Wie kann ich einen Literatur-Typen anlegen/bearbeiten?}
\begin{enumerate}
\item gehe auf \textit{http://localhost:8081/overview}
\item Klicke auf \textit{Neuen Typen erstellen}\newline
\includegraphics[width=\linewidth]{dokuImages/gui1_1.png}
\item gib einen Namen für den Literaturtyp an\\
\includegraphics[width=\linewidth]{dokuImages/gui1_2.png}
\item Attribute fürs Literaturverzeichnis eingeben und stylen\\
\includegraphics[width=\linewidth]{dokuImages/gui1_3.png}
\item Attribute für Zitate auswählen\\
\includegraphics[width=\linewidth]{dokuImages/gui1_4.png}
\item Speichern
\item zum Bearbeiten oder Löschen einfach auf die Icons in der Liste klicken
\end{enumerate}

\subsection{Wie kann ich einen Literatur-Eintrag anlegen/bearbeiten?}
\begin{enumerate}
\item gehe auf \textit{http://localhost:8081/overview}
\item Klicke auf \textit{Neuen Eintrag erstellen}\newline
\includegraphics[width=\linewidth]{dokuImages/gui2_1.png}
\item gib einen Key ein, wähle einen Typen aus und fülle die Felder aus (die Felder passen sich an den Typen an)\\
\includegraphics[width=\linewidth]{dokuImages/gui2_2.png}
\item Speichern
\item zum Bearbeiten oder Löschen einfach auf die Icons in der Liste klicken
\end{enumerate}

\subsection{Wir kann ich diesen erstellten Eintrag dann zitieren und im Literaturverzeichnis aufnehmen?}
Im Literaturverzeichnis wird automatisch alles aufgelistet.\\[6pt]Um die Quelle zu zitieren kann folgender Befehl im Text aufgerufen werden:
\begin{verbatim}
\citebib{KEY}{SEITEN}{Vgl.}
\end{verbatim}
Der zweite und dritte Parameter können leer bleiben.

\part{Wie wurde dieses Template umgesetzt?}

\part{Häufig auftretende Fehler/Warnungen}
\section{Overfull hbox}
\subsection{Silbentrennung}
\section{Invalid Character, missing \$ inserted, o.ä.}
Viele Sonderzeichen werden von LaTeX für Befehle verwendet und können deshalb nicht einfach so im Text verwendet werden. Man kann am besten googeln oder ausprobieren, einfach ein \textbackslash{} vor das Sonderzeichen zu setzen.
\section{Leerzeichen nach Sonderzeichen fehlt}

\section{Text eingeschoben}

\section{Bilder und Tabellen fließen in Text}

\clearpage
\frontmatter%Stil des Headers/Footers ändern
\renewcommand{\plaintitle}{Literaturverzeichnis}
\pagenumbering{Roman}
\setcounter{page}{5}
\addtocontents{toc}{\vspace{24pt}}
\addcontentsline{toc}{part}{Literaturverzeichnis}%Literatur-Verz. ins Inhaltsverzeichnis
\printMyBibliography

\end{document}