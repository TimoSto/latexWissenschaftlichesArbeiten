Es bestehen verschiedene Möglichkeiten den negativen Einfluss von Kryptowährungen zu verringern, sowohl auf Seiten der Regierungen, als auch durch Unternehmen und die Wahl der Kryptowährung seitens Anwendern. Zwar ist Bitcoin und das verwendete Proof-of-Work Verfahren aktuell noch am bekanntesten, doch es bieten sich viele Alternativen, die nicht nur im Bezug auf Energieeffizienz, sondern auch in den Bereichen Sicherheit und Fairness Chancen bieten den Status quo zu verbessern. Trotzdem existieren gerade bezüglich alternativer Konsensverfahren noch Diskussionen im Bezug auf die Sicherheit, vor allem da viele Verfahren noch nicht in dem Maß getestet und erprobt sind wie das Proof-of-Work Verfahren. Solange diese Alternativen zum Proof-of-Work-Verfahren noch nicht ausgereift sind, ist ein ein Wechsel vom Bargeldsystem zu einer Kryptowährung aus ökologischer Sicht nicht zu empfehlen. Während eine Senkung der verbrauchten Energie also dringend gefordert wird, müssen erste Schritte seitens der Länder, Unternehmen und Einzelpersonen ergriffen werden, um einen Wechsel zu ermöglichen und durch Forschungsdaten die Folgen von alternativen Verfahren zu beurteilen.