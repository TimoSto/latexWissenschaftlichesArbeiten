Seinem Namen entsprechend fordert der Proof-of-Burn Konsensmechanismus von Minern das „Verbrennen“ von Kryptowährung als seinen Einsatz. Das sogenannte „Verbrennen“ beschreibt dabei den Transfer von Coins zu einer Adresse, auf die kein Zugriff besteht, auch nicht seitens des Senders\citebib{verma}{}{vgl. }. Dorthin verschickte Coins können also auch nicht mehr eingesetzt werden und werden dem System gewissermaßen entzogen bzw. „zerstört“.\\
Beim Proof-of-Burn verbrennen Miner auf diese Weise einen Teil ihrer Coins, meist die einer anderen Kryptowährung, um sich für die Validierung als vertrauenswürdig zu erweisen. Je höher dabei die Menge der über die Zeit zerstörten Coins ist desto höher ist auch die Chance für die Validierung des nächsten Blocks ausgewählt zu werden\citebib{he}{}{vgl. }. Um den Verlust von Coins langfristig auszugleichen, werden dabei oftmals Belohnungen für jeden neuen Block ausgegeben.
Ein Vorteil dieses Konsensmechanismus ist, dass er sich nicht auf den Einsatz von Rechenleistung oder Hardware verlässt und somit abgesehen von den verbrannten Coins einen geringeren Aufwand verursacht\citebib{verma}{}{vgl. }. Die Sicherheit wird durch den hohen Einsatz in Form von Coins gewährleistet. Indem die Miner eine hohe Anfangsinvestition leisten, sollen sie so langfristig ehrlich und am System interessiert bleiben, um diese stetig zurückzuerlangen\citebib{hazari}{}{vgl. }.