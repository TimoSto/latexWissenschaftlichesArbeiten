Eine Möglichkeit, um negative Auswirkungen auf die Umwelt durch Kryptowährungen zu verringern, ohne dass der generelle Stromverbrauch dabei gesenkt werden muss, ist das Beziehen des Stroms aus erneuerbaren Quellen und somit die Verwendung von „grünem Strom“.\\
Kryptowährungen bedienen sich meist Blockchain-Verfahren und nutzen die Vorteile des damit erbauten dezentralen Systems. Gerade diese Dezentralisierung erschwert allerdings auch die Ermittlung von aktuellen Werten zu verwendeten Stromquellen. Für eine offene, nicht zugangsbeschränkte Proof-of-Work-Blockchain ist bereits der Stromverbrauch nur schwer bestimmbar, da „[…] man in der Regel weder die beim Mining eingesetzte Rechenleistung noch die entsprechende Hardware für jeden Teilnehmer bestimmen kann.“\citebib{infoSpektrum}{S.394}{}. Der Standort, sowie weitere Faktoren wie der technische Stand der Ausrüstung und das vorherrschende Klima, beeinflussen wiederum den Energieverbrauch und die CO2-Emissionen. Diese Faktoren in Kombination mit fehlenden Daten über Miner und deren Ausrüstung führen zu einer Unsicherheit in den ermittelten Werten zum Stromverbrauch und Umwelteinfluss, die sich in auseinandergehenden Kennzahlen niederschlägt. Beispielsweise werden laut einer Studie von Cambridge bereits durchschnittlich 39\% des Proof-of-Work Mining mit erneuerbaren Energien durchgeführt\citebib{blandin}{S.26}{vgl. }. Andere Studien, wie die der CoinShares Research, gehen sogar von einem Prozentanteil von 74\% beim Bitcoin-Mining aus\citebib{bendiksen}{}{vgl. }.\\
Neben den genannten Faktoren spielt auch die ständige Weiterentwicklung und Veränderung der Hauptstandorte von Mining-Farmen und Minern eine Rolle. Ein Beispiel dafür ist das 2021 erlassene Verbot der chinesischen Regierung gegen Bitcoin, durch das China von einem Anteil von geschätzten 46,04\% an der Netzwerk-Hash-Rate im April 2021 auf mittlerweile 0\% gefallen ist. Andere Länder, wie beispielsweise Amerika, profitierten anschließend von einem Zuwachs an Minern\citebib{btcDist}{}{vgl. }. Durch eine solche Verschiebung der Anteile bzw. Standorte ändert sich ebenfalls die Zusammensetzung der für das Mining verwendeten Energiequellen, die ihrerseits abhängig von den Bezugsquellen des jeweiligen Landes sind.\\
Im Allgemeinen sind die Aussichten auf eine Senkung der CO2-Emissionen durch „grünen Strom“ durchaus positiv einzuschätzen, da eine Verwendung von erneuerbaren Energien sich für die Miner als wirtschaftlich lohnend zeigt. Insbesondere bei dem Konsensmechanismus des Proof-of-Work, bei dem die Wahrscheinlichkeit den nächsten Block schreiben zu dürfen mit der Rechenleistung steigt, muss der Ertrag durch durchschnittlich gewonnene Einheiten der Kryptowährung größer sein als der Aufwand durch den verwendeten Strom und gegebenenfalls auch angeschaffte Hardware. Je höher die Differenz zwischen Ertrag und Aufwand dabei ist, desto höher fällt auch der Gewinn aus, sodass ein möglichst günstiger Strompreis essentiell für die Höhe des Gewinns ist. Strom aus erneuerbaren Energien bietet sich hierbei an, da dieser oftmals günstiger und somit wirtschaftlich vorteilhafter für die Miner ist\citebib{irena}{}{vgl. }.\\
Zu beachten ist allerdings die Abhängigkeit vom Standort und in die dortige Verfügbarkeit von Strom aus erneuerbaren Energien. Sollten Miner von einem Land aus agieren, dass sich beispielsweise hauptsächlich auf Strom aus Kohlekraftwerken stützt, müssten sie sich dem für sie zugänglichem Strom bedienen, oder wären gezwungen ihren Standort zu wechseln. Diese Zuständigkeit der Länder bietet sowohl Vor- als auch Nachteile. Zwar gewinnen diese an Entscheidungsmacht und können Maßnahmen unabhängig von den Systemen der Kryptowährungen vornehmen, allerdings gehen diese Maßnahmen nicht die Ursache des Problems – allen voran der Konsensmechanismus – an, sondern dämmen lediglich die Schäden ein. Zudem könnte eine Entwicklung, durch die Strom aus umweltschädlichen oder nicht erneuerbaren Energien günstiger wird, dazu führen, dass Miner dem Prinzip der Gewinnmaximierung folgend diese Variante wählen.
