Da viele der Kryptowährungen die Blockchain-Technologie nutzen und in einem solchen dezentralen Netzwerk eine Notwendigkeit besteht sich auf bestimmte Status oder Datenwerte zu einigen, werden Konsensverfahren eingesetzt. Eines der bekanntesten Verfahren, das in Kapital 2.4.2 bereits erläutert wurde ist dabei das Proof-of-Work Verfahren. Wie die meisten Konsensverfahren fordert auch Proof-of-Work eine Art von Einsatz, um an der Auswahl zum Schreiben des nächsten Blocks teilzunehmen. Proof-of-Work nutzt hierbei die Rechenleistung als seine Art von Einsatz, was in der Kombination mit der Konkurrenz um jeden Block zu dem hohen Energieverbrauch führt. Allerdings gibt es weitere, alternative Konsensverfahren, die mit einem anderen Einsatz arbeiten und somit den Energieverbrauch und auch die Verschwendung von Energie senken.\\
Die Kryptowährung Ether, die bisher ebenso wie Bitcoin den Proof-of-Work Mechanismus verwendet hat, befindet sich beispielsweise aktuell in einer Umstellung auf den Proof-of-Stake-Mechanismus. Auch andere Mechanismen könnten eine umweltschonende Alternative zu dem Status quo bieten. Einige solcher alternativen Mechanismen sollen im Folgenden genauer betrachtet werden.