Der Proof-of-Space-Mechanismus, oder auch Proof-of-Capacity genannt, fordert als seine Art von Einsatz Festplattenspeicher. Ein einzelner “Proof-of-Space” bezeichnet dabei Daten, die belegen, dass ein Anwender eine bestimmte Menge an Speicherplatz reserviert hat\citebib{he}{}{vgl. }. Was folgt ist ein Vorgang der im Rahmen des Burstcoin Netzwerks auch als Plotting bezeichnet wird. Berechnungen werden einmalig durchgeführt und die Ergebnisse anschließend gespeichert, sodass sie später beim Mining schlichtweg ausgelesen werden können. Die Wahrscheinlichkeit für die Validierung ausgewählt zu werden steigt dabei proportional mit dem bereitgestellten Speicher. Auf diese Weise wird der Energieverbrauch im Vergleich zum Proof-of-Work Verfahren gesenkt, allerdings könnte der Einsatz dazu führen, dass die Konkurrenz um Rechenleistung sich zu einer Konkurrenz um Speicherplatz entwickelt\citebib{hazari}{}{vgl. }.