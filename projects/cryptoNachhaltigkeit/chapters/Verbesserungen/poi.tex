Das Proof-of-Importance Verfahren ähnelt dem Proof-of-Stake Verfahren und ist gleichzeitig ein Ansatz, um dem in letzterem festgestellten Problem der Bevorzugung von reichen Personen entgegenzuwirken. Hierzu werden neben dem Vermögen auch die Haltedauer und die Aktivität der Person zu einem sogenannten Importance-Score verrechnet. Auf diese Weise sollen Personen gefunden werden, die besonders wichtig für das System sind und somit ein hohes Interesse an dem Fortbestehen und der Korrektheit des Systems haben. Teilnehmer, die sehr aktiv sind, also viele Transaktionen durchführen, haben eine höhere Chance als Validator für den nächsten Block ausgewählt zu werden\citebib{hazari}{}{vgl. }.\\
Durch die Auswahl eines Validators entfallen auch hier die Aufwände der Konkurrenz und die Notwendigkeit für spezielle technische Ausrüstungen, um an der Auswahl teilnehmen zu können. Allerdings führt die Verwendung des Proof-of-Importance Verfahrens auch dazu, dass es für neue Teilnehmer zunächst schwierig ist als Validator ausgewählt zu werden und auch eine lange Abwesenheit bzw. Inaktivität wirkt sich negativ auf die Chance Validator zu werden aus.