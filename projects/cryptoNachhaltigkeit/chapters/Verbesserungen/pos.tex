Das Proof-of-Stake Verfahren ist einer der bekanntesten Konsensmechanismen neben dem Proof-of-Work und agiert über eine zufällige, aber gewichtete Auswahl des Validatoren. Das für die Gewichtung betrachtete Kriterium ist dabei der eingesetzte Stake, eine bestimmte Menge von Kryptowährung, die als Einsatz genutzt wird. Abhängig von der Höhe des Stakes wird so ein Validator bestimmt, der den nächsten Block schreiben darf und die dafür angesetzte Transaktionsgebühr bekommt\citebib{hazari}{}{vgl. }.\\
Durch den Wegfall der Konkurrenz ist der Energieverbrauch im Vergleich zum Proof-of-Work Verfahren um einiges geringer und auch die Notwendigkeit zu spezieller Ausrüstung, um an der Auswahl teilzunehmen, entfällt.\\
Allerdings verschafft das Proof-of-Stake Verfahren Personen, die bereits viele Einheiten der Kryptowährung besitzen einen Vorteil, sodass „reiche“ Personen eine höhere Chance haben ausgewählt zu werden. Um dem entgegenzuwirken können weitere Faktoren zur Auswahl mit einbezogen werden, beispielsweise die Dauer, über welche eine Person bereits im Besitz seiner Menge an Kryptowährung ist oder wie lange er seinen Stake bereits gesetzt hat\citebib{hazari}{}{vgl. }.\\
Ein weiterer zu berücksichtigender Faktor ist die Sicherheit, die das Proof-of-Stake Verfahren bietet. Validiert jemand einen manipulierten oder gefälschten Block und wird dabei erwischt, wird demjenigen ein Teil bzw. sein gesamter Stake entzogen, sodass Teilnehmer zur Aufrichtigkeit angehalten werden. Abgesehen von diesem Aspekt gibt es jedoch Bedenken bzgl. der konzeptionellen und technischen Sicherheit des Proof-of-Stake-Verfahrens. Im Vergleich zum Proof-of-Work-Verfahren ist dieses anfälliger für bestimmte Angriffe und bringt somit ein höheres Risiko mit sich\citebib{nair}{}{vgl. }. Ausgenommen hiervon ist allerdings der sogenannte 51\%-Angriff. Durch einen solchen Angriff schafften es Hacker bei Bitcoin Gold Coins in einem Wert von bis zu 18 Millionen Dollar zu erbeuten, indem sie Kryptowährung doppelt ausgaben. Ein solcher Angriff würde in einem System mit dem Proof-of-Stake Verfahren erheblich erschwert, da zur Durchführung 51\% der Coins nötig wären und ein Angreifer Gefahr stünde, diese durch den Angriff zu verlieren\citebib{suhyeon}{}{vgl. }. Jedoch gibt es auch hier bereits Theorien, laut denen ein solcher Angriff auf ein Proof-of-Stake System nicht nur durchführbar, sondern auch potenziell lohnend für den Angreifer sein und das gesamte Proof-of-Stake Kryptowährungssystem eventuell sogar zerstören könnte (vgl. ebd.)\citebib{houy}{}{vgl. }.