Wie in den vorangegangenen Kapiteln erläutert wurde, verursachen die mit Kryptowährungen wie beispielsweise Bitcoin verbundenen Prozesse, insbesondere das Mining, einen hohen Energieaufwand. Der verbrauchte Strom ist allerdings nicht der einzige Grund, aus dem sich in den letzten Jahren immer mehr Bedenken zu den möglichen Auswirkungen auf die Umwelt geäußert haben. Auch die dabei entstehenden CO2-Emissionen, sowie die Menge an elektronischem Abfall sind zu berücksichtigen. Letztere entsprachen allein im Fall von Bitcoin bereits der Menge des Landes Luxemburg\citebib{badea}{}{vgl. }. Hinzu kommen CO2-Emissionen, die laut einer Schätzung in dem Zeitraum vom 01.01.2016 bis zum 30.06.2018 für die Kryptowährungen Bitcoin, Ethereum, Litecoin und Monero bei drei bis fünfzehn Tonnen gelegen haben sollen\citebib{krause}{}{vgl. }. Eine weitere Studie der Nature Climate Change wiederum ergab, dass bei einer ähnlichen Entwicklung von Bitcoin zu anderen Technologien, Bitcoin allein bis 2033 eine Erhöhung der globalen Temperatur um 2 Grad Celsius verursachen könnte\citebib{mora}{}{vgl. }.\\
Insbesondere aufgrund der letztgenannten Studie, die den von Cambridge ermittelten fatalen Wert von 1,5 Grad Celsius übersteigt\citebib{ipcc}{}{vgl. }, werden deshalb Maßnahmen empfohlen, um den Energieverbrauch in Zukunft zu senken. Vereinfacht gibt es hierbei zwei Punkte an denen angesetzt werden könnte - die Quelle des verwendeten Stroms, und in dieser Hinsicht die Betrachtung von „grünem Strom“, und zum anderen der etablierte und bei Kryptowährungen oftmals verwendete Konsensmechanismus des Proof-of-Work. Der Proof-of-Work Mechanismus fordert nicht nur einen hohen Energieaufwand, sondern führt seinem Prinzip nach bereits zu einem gewissen Maß an Verschwendung. Indem die Möglichkeit den nächsten Block zu validieren dem schnellsten Miner zugesprochen wird, bleibt die aufgewandte Rechenleistung der restlichen Miner fruchtlos.