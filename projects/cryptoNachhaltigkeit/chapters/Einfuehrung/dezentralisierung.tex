Der erste wichtige Baustein und eine der bedeutendsten Eigenschaften von Kryptowährungen ist die Dezentralisierung. Das bedeutet, dass weder der Handel, noch die Ausgabe von Kryptowährungen zentral von einem Staat oder von einer Bank abgewickelt werden, was bei herkömmlichen Währungen hingegen der Fall ist. Die Ausgabe und der Handel finden ausschließlich im Internet unter den Nutzern direkt statt und dies sorgt so für eine Unabhängigkeit von übergeordneten Instanzen, welche sich sonst um die Verwaltung kümmern müssten\citebib{neumann}{S.4}{vgl. }.\\
Die technische Grundlage hierfür bildet ein Peer-to-Peer Netzwerk (P2P). Peer-to-Peer Netzwerke sind Rechnernetze, bei denen alle Rechner im Netz miteinander verbunden sind und gleichberechtigt miteinander arbeiten\citebib{hoenig}{S.35}{vgl. }. Jeder Rechner kann Funktionen und Dienstleistungen anbieten und von anderen Rechnern angebotene Funktionen, Ressourcen, Dienstleistungen und Dateien nutzen. Zudem kann jeder Rechner mit mehreren anderen Rechnern verbunden sein. Die Daten sind dabei in der Regel auf den Rechnern der Nutzer verteilt. Es handelt sich hier demzufolge um ein dezentrales Konzept, da es keinen übergeordneten, zentralen Server zur Verwaltung gibt\citebib{neumann}{S.6}{vgl. }.\\
Das Peer-to-Peer Netzwerk von Kryptowährungen basiert auf einer von allen Nutzern gemeinsam genutzten und verwalteten (also dezentralen) Datenbank, in der alle Transaktionen zwischen den Nutzern überwiegend in einer sogenannten Blockchain gespeichert werden. Diese Art der Datenverwaltung wird auch Distributed-Ledger-Technologie, kurz Ledger genannt. Das Netzwerk bzw. der Ledger fungiert dabei wie ein globales, kollektives Buchführungssystem mit dem Zweck, das Verdoppeln oder Fälschen der Kryptowährung zu verhindern\citebib{neumann}{S.7}{vgl. }.\\
Im Gegensatz zu einem klassischen Kassenbuch, wo die Daten auf einem einzigen System gespeichert und von einer zentralen Instanz gepflegt werden, sind die Informationen bei einer Blockchain auf eine große Anzahl von Netzwerkteilnehmern, sogenannte „Knoten“, verteilt. Die Knoten vertrauen sich gegenseitig nicht und das Kassenbuch wird mehrfach abgespeichert. Das hat zur Folge, dass eine Vielzahl von Computern weltweit identische Kopien der gesamten Transaktionshistorie besitzen. Diese dezentrale Art der Datenorganisation hilft dabei, Angriffe auf das Netzwerk zu verhindern und das Vertrauen in die erfassten Daten zu erhöhen\citebib{schuster}{S.4}{vgl. }.