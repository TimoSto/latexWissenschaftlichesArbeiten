Zur Verwendung der digitalen Währung besitzt jeder Nutzer eine sogenannte Wallet (Portemonnaie), wodurch es ihm möglich ist, sein Konto zu verwalten. Die hierfür nötige Clientsoftware wird durch verschiedene Anbieter im Internet vertrieben. Anders als bei einer herkömmlichen Wallet wird hier jedoch nicht direkt das Geld aufbewahrt. Stattdessen befinden sich hier die beiden Schlüssel zur eigenen Bitcoin-Adresse. So gibt es für jeden Benutzer ein Schlüsselpaar aus privatem und öffentlichem Schlüssel, welche beide zusammenpassen. Mit dem öffentlichen Schlüssel wird die eigene Bitcoin-Adresse für alle Clients im Netzwerk sichtbar\citebib{neumann}{S.7}{vgl. }. Durch den privaten Schlüssel wiederum kann der Absender seine Transaktion verschlüsseln – man spricht hier auch von „signieren“. Jeder beliebige Netzwerkteilnehmer kann nun die verschlüsselte Transaktion verifizieren, indem er diese mit dem öffentlichen Schlüssel entschlüsselt und so sicherstellen kann, dass sie tatsächlich vom Absender stammt. Auf diese Weise werden Authentizität und Integrität der einzelnen Transaktionen gewährleistet, denn man kann eine Transaktion ihrem Absendern zuordnen und gleichzeitig sicherstellen, dass sie nicht manipuliert wurden\citebib{schuster}{S.5f.}{vgl. }.\\
Wenn ein Nutzer eine Transaktion ausführen möchte, muss er mithilfe seines privaten Schlüssels seine Nachricht signieren. Die Nachricht wiederum beinhaltet dann die Informationen wie den Sender, den Empfänger und die Höhe des zu versendenden Betrags. Sobald die Nachricht ordnungsgemäß mit dem privaten Schlüssel signiert wurde, wird die Information des Versendens der Transaktion an alle Nutzer geschickt. Wenn eine gewisse Zeit vergangen ist (bei Bitcoin beispielsweise ungefähr 10 Minuten), werden die getätigten Transaktionen in einem Block zusammengefasst, von den Minern verifiziert und anschließend dem öffentlichen Transaktionsprotokoll, der sogenannten Blockchain, hinzugefügt\citebib{neumann}{S.11f.}{vgl. }. Miner könnte man als die „Verwalter“ des Systems bezeichne. Es handelt sich hier um eine Gruppe von Menschen, die großes Interesse an der korrekten Funktionsweise der Kryptowährung hat, aber in der Regel nicht in Kontakt zueinander stehen.