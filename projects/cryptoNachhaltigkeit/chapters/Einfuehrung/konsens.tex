Da es keine Möglichkeit gibt, bereits existierende Informationen abzuändern oder gar zu löschen, ist eine hohe Sicherheit des richtigen Hinzufügens der Transaktionsblöcke ein zentrales Element für die Funktion von Kryptowährungen. Grundsätzlich haben viele oder sogar jeder Knoten die Möglichkeit, einen Block hinzuzufügen. Es muss daher geregelt werden, welcher dieser Knoten den nächsten Block in die Blockchain übertragen darf. Aufgrund des dezentralen Aufbaus des System gibt es jedoch keine zentrale Instanz, die diese Aufgabe übernehmen könnte. Diese Rolle übernehmen Konsensmechanismen\citebib{schuster}{S.6}{vgl. }.\\
Um möglichst viele Knoten (Miner) für die Arbeit zu finden, wird diesen eine Belohnung für das Hinzufügen eines neuen Blockes in Aussicht gestellt. Die Miner stehen folglich in Konkurrenz zueinander und es ist wichtig, diesen Wettbewerb sinnvoll zu gestalten. So sollten die Knoten beispielsweise einen Anreiz haben, nur Blöcke mit legitimen Transaktionen zu erstellen, die weder mit sich selbst, noch mit bereits auf der Blockchain vorhandenen Transaktionen in Konflikt stehen\citebib{schuster}{S.6}{vgl. }.