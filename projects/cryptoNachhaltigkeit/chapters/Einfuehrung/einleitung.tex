Mit der Einführung des Internets ist es möglich geworden, auf einfachem Wege weltweit mit anderen Menschen zu kommunizieren und Informationen auszutauschen. Diese Gegebenheit ist unter anderem dazu genutzt worden, eine globale Handelsplattform unter den Nutzern zu schaffen. So haben sich mit der Zeit verschiedenste Onlineshops gebildet, die es ermöglichen, Güter und Waren aus der ganzen Welt zu erwerben. Ein Hindernis für diese globale Handelsmethode stellen jedoch die weltweit unterschiedlichen Währungen dar, da diese keinen einheitlichen Zahlungsverkehr ermöglichen. Im Jahre 2009 kam es zu einer ersten Idee, dieses Problem mittels einer einheitlichen Währung zu lösen. Mit der Einführung des Bitcoins, der ersten Kryptowährung, wurde ein globales, dezentrales Zahlungsmittel geschaffen, welches ausschließlich im Internet gehandelt werden konnte und keinen physikalischen Wert in der realen Welt wie herkömmliches Geld besitzt\citebib{neumann}{S.2}{vgl. }. Die Schaffung des Bitcoins hatte mit ihrer Idee einer Kryptowährung die Geburt vieler weiterer Kryptowährungen zur Folge. In den folgenden Unterkapiteln wird nun grob darauf eingegangen, welche Eigenschaften diese Kryptowährungen haben und wie das Ganze technologisch (am Beispiel Bitcoin) in der Praxis umgesetzt wird.