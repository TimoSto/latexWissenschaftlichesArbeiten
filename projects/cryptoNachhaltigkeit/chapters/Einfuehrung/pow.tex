Der am meisten verbreitete Konsensmechanismus ist der Proof-of-Work-Mechanismus, welcher u.a. dem Bitcoin-Netzwerk zugrunde liegt. Die Kernidee besteht darin, die Erstellung eines Blocks teurer zu machen, die Überprüfbarkeit der Korrektheit eines Blockkandidaten wiederum aber sehr günstig für alle anderen zu gestalten. „Falschmeldungen“ werden hierdurch teuer und können gleichzeitig leicht entlarvt werden. Dabei werden ungültige Blöcke von „ehrlichen“ Knoten einfach ignoriert, wohingegen ein neuer gültiger Blockkandidat Teil der Blockchain wird, indem die Miner auf diesen Block bei ihrer nächsten Blockerstellung referenzieren\citebib{schuster}{S.7}{vgl. }.\\
In einem PoW-Netzwerk wird die Berechtigung, einen neuen Block erstellen zu können, an die schnellstmögliche Lösung einer kryptografischen Aufgabe geknüpft. Die Belohnung für die dafür eingesetzte Rechenleistung sind eine Menge an Coins der jeweiligen Kryptowährung. Diese Belohnung erhält aber nur der Miner, welcher die Lösung zuerst findet. Folglich stehen die Miner untereinander im Wettbewerb und investieren möglichst viel Rechenleistung, um den nächsten Block erstellen zu dürfen\citebib{kaucher}{S.3f.}{vgl. }.  Gerade dieser hohe Energieverbrauch ist bei PoW-Netzwerken als Schutzmechanismus gegen Angriffe und Datenmanipulation gedacht\citebib{sedlmeir}{S.600f.}{vgl. }.\\
Es kann jedoch passieren, dass zeitgleich zwei korrekte neue Blöcke der Blockchain hinzugefügt werden, so dass gleichzeitig zwei Versionen der Blockchain im Umlauf sind. Um den aktuellen Zustand festzulegen gibt es die Konsensregel, dass die Kette mit der höchsten eingeflossenen Arbeitsleistung (also tendenziell die längste Kette) die Aussagekräftige ist. Umso mehr Blöcke also auf einen bestimmten Block folgen, desto sicherer ist es, dass dieser Block als Teil der Blockchain vom Kollektiv betrachtet wird. So erachtet man beispielsweise im Bitcoin-Netzwerk Blöcke mit fünf oder mehr Nachfolgeblöcken als sicheren Teil der Blockchain\citebib{schuster}{S.7f.}{vgl. }.\\
Zusammengefasst erschwert der Proof-of-Work Mechanismus das Hinzufügen neuer Blöcke künstlich, indem zunächst eine geeignete Nonce gefunden werden muss. Die Schwierigkeit im Bitcoin-Netzwerk wird zum Beispiel so gewählt, dass im Schnitt alle zehn Minuten ein neuer Block entsteht. Durch diesen Mechanismus wird geregelt, dass sich das Kollektiv an Netzwerkteilnehmern auf den aktuellen Zustand der Blockchain einigt und den darin enthaltenen Daten vertraut. Die Kosten für einen erfolgreichen Angriff auf die Daten (die hohe benötigte Rechenleistung) sind extrem hoch, während gleichzeitig Anreize für ehrliches Verhalten gesetzt werden. Dadurch wird ein vertrauenswürdiger Intermediär bei der Blockchain durch technologisches Vertrauen ersetzt\citebib{schuster}{S.7f.}{vgl. }.