Alle getätigten Transaktionen innerhalb eines Netzwerks für Kryptowährungen werden aufgezeichnet und in einer Blockchain gespeichert. Es handelt sich hierbei um eine verteilte Open-Source-Datenbank, welche auf moderner Kryptografie basiert. Die Blockchain übernimmt zwei zentrale Aufgaben: sie bietet eine Methode zum Organisieren und Speichern von Daten und gleichzeitig eine Methode, um das Vertrauen in die Daten zu fördern\citebib{schuster}{S.3f.}{vgl. }.\\
Bei einer Blockchain werden die Informationen in einer stetig wachsenden Liste von Datensätzen organisiert. Wichtig ist hier, dass lediglich Daten hinzugefügt, aber frühere Daten nicht verändert oder gelöscht werden können. Bei Änderungen werden alte Einträge also nicht überschrieben, sondern neue Einträge erzeugt. Auch wenn diese Methode vergleichsweise umständlich und speicherintensiv ist, wird durch die Dokumentation der gesamten Historie die Transparenz und Betrugssicherheit stark gefördert. Die Transaktionsdaten werden dabei aus Effizienzgründen in Blöcken zusammengefasst\citebib{schuster}{S.4}{vgl. }.\\
Die Blockchain selbst besteht aus einer Aneinanderreihung dieser Blöcke (daher der Name „Blockchain“), wobei jeder Block die Transaktionsdaten von einem bestimmten Zeitraum beinhaltet. Dadurch fungiert die Blockchain als eine Art globales Kontobuch für das System. Ein sogenannter Miner hat dabei die Aufgabe, die Transaktionen eines Blocks zu bestätigen, damit man diesen wiederum zu der bestehenden Kette hinzufügen kann. Für das Erledigen der Aufgabe erhält der Miner eine Belohnung in Form der Währung des Systems\citebib{neumann}{S.8}{vgl. }.\\
Ein Block einer Blockchain besteht dabei aus drei Dingen: den Daten (im Fall von Kryptowährung den Transaktionsdaten), seinem eindeutigen Hash-Wert und dem Hash-Wert des vorherigen Blocks. Hash-Werte sind eine Aneinanderreihung von Zahlen und Buchstaben und werden mithilfe von Hash-Funktionen gebildet. Diese Hash-Funktionen sorgen dafür, dass eine große Datenmenge (wie zum Beispiel der gesamte Inhalt eines Blocks) auf eine kleine Datenmenge (den Hash-Wert) festgelegter Größe (beispielsweise 256 Bit bei Bitcoin) abgebildet wird. Dabei gibt zum Beispiel die Bitcoin-Blockchain den Maximalwert für den zu ermittelnden Hash-Wert durch die Anzahl der führenden Nullen vor. Um den Zielwert zu erreichen, wird einem Block eine Zahl (sogenannte Nonce) hinzugefügt. Diese Zahl wird nun durch Ausprobieren so lange verändert, bis der Hash-Wert des Blocks die Vorgabe erfüllt. Die Besonderheit der hier verwendeten kryptografischen Hash-Funktionen ist, dass es keine bessere Möglichkeit gibt, als durch Ausprobieren den gesuchten Wert zu ermitteln, da keine direkte Berechnung möglich ist. Die Anzahl der Versuche pro Zeiteinheit, eine zulässige Lösung zu finden, wird als Hash-Rate bezeichnet. Nach seiner Ermittlung wird der Hash-Wert im Block aufbewahrt und dieser Block wiederum wird am Ende der Blockchain gespeichert\citebib{schuster}{S.6f.}{vgl. }.\\
Zur vereinfachten Darstellung kann die Bedingung für einen validen Block als Funktion der Daten und des vorherigen Hashes sowie der Nonce dargestellt werden\citebib{kaucher}{S.2}{vgl. }:
\[Hash_{(\text{Blockdaten}, \text{vorheriger Hash}, \text{Nonce})}\in \text{Hashes mit Hash-Bedingung}\]
Jeder Hash ist dabei einzigartig und vom restlichen Inhalt des Blocks abhängig, man kann ihn sich also als eine Art Fingerabdruck des Blocks vorstellen. Wird ein Zeichen im Block verändert, verändert sich gleichzeitig auch sein Hash-Wert. Ändert sich der Hash-Wert, handelt es sich nicht mehr um den gleichen Block. Wichtig ist dabei, dass alle Daten des Blocks bei der Erstellung des Hash-Werts von den Minern beachtet werden. Dazu gehören somit nicht nur die Transaktionsdaten, sondern auch zusätzliche Daten wie der Hash-Wert des letzten Blocks. Durch diesen Mechanismus besitzt jeder neue Hash also Daten des vorherigen Blocks und kann diesen somit bestätigen. Das führt zu einer hohen Sicherheit des Systems\citebib{neumann}{S.8}{vgl. }.\\
So würde eine manipulative Änderung in einem beliebigen Block der Kette auch den Hash-Wert dieses Blocks ändern. Folglich würden sich auch die Hash-Werte aller nachfolgenden Blöcke verändern, weil sich dadurch ihre Referenzen auf den vorherigen Block und wiederum deswegen auch ihr eigener Hashwert ändert. Eine solche Manipulation ist also äußerst aufwändig und bei Fehlern im Vorgehen leicht zu erkennen\citebib{schuster}{S.7}{vgl. }.