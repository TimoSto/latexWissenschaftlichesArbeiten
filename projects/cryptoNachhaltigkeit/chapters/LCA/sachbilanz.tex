Zu der Sachbilanz gehören Datenerhebungen und Berechnungsverfahren, welche verwendet werden, um Input und Output eines Produktsystems zu quantifizieren. Bei der Datenerhebung werden die Daten in Hauptgruppen unterteilt. Diese Hauptgruppen können zum Beispiel Energie-Inputs, Rohstoff-Inputs, Abfall, Emissionen oder weitere umfassen. Nach der Datenerhebung werden Berechnungsverfahren verwendet, um die gesammelten Daten zu validieren und den Prozessmodulen oder funktionellen Einheiten zuzuordnen\citebib{iso2009}{, S.25ff.}{vgl. }.