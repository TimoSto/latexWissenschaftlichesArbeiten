Die Ergebnisse der Sachbilanz werden in der nächsten Phase, der Wirkungsabschätzung, verwendet, um die potenziellen Umweltwirkungen des Produktes zu beurteilen. Hierbei werden Kategorien und Indikatoren verwendet, um die Auswirkungen zu erkennen. Die Wirkungsabschätzung kann auch verwendet werden, um festzustellen, ob das Ziel der Studie erreicht wurde\citebib{iso2009}{, S.25ff.}{vgl. }.\\
Bei der Auswahl einer Wirkungsabschätzungsmethode kann zwischen zwei grundlegenden Ansätzen unterschieden werden. Der erste ist der sogenannte „Midpoint“-Ansatz, welcher Problem-orientiert ist. Bei dieser Methode werden die Wirkungsindikatoren nah an der Ursache definiert. Das Gegenstück zu diesem Ansatz ist der „Endpoint“-Ansatz. Die Wirkungsindikatoren bei diesem Ansatz beschreiben meist den entstehenden Schaden, weshalb ihre Relevanz für die Umwelt höher ist als beim „Midpoint“-Ansatz. Eine Methode, welche beide Ansätze verbindet, ist die ReCiPe-Methode\citebib{bruijn}{S.530}{vgl. }.