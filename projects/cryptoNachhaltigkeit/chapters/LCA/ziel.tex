In der ersten Phase des Life Cycle Assessments werden zunächst das Ziel und der Untersuchungsrahmen definiert. Das Ziel beinhaltet die beabsichtigte Anwendung und die Gründe für die Durchführung der Studie. Weiterhin gibt es an, welche Zielgruppe angesprochen werden soll und ob die Ergebnisse zur Veröffentlichung bestimmt sind. Um sicherzustellen, dass der Detaillierungsgrad der Studie für das Ziel ausreichend ist, muss der Untersuchungsrahmen hinreichend definiert sein. Dieser beinhaltet das zu untersuchende Produktsystem und seine Funktion, die Systemgrenze, die Anforderungen an die Datenqualität und weitere Aspekte. Durch die iterative Herangehensweise kann der Untersuchungsrahmen sich im Laufe der Studie verändern. Dies kann zum Beispiel durch Erkenntnisse in der Sachbilanz erforderlich werden \citebib{iso2009}{, S.22ff.}{vgl. }.
