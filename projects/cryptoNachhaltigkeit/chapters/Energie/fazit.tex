Das Grundkonzept von Blockchains, vor allem die Dezentralisierung, führen zu einer gewissen Redundanz. Durch diese und den Rechenauswand für das Erstellen und Validieren von neuen Blocks entstehen zwar immer Energieaufwände, diese sind aber nicht zwangsläufig problematisch. Die Proof-of-Work-Netzwerke haben einen sehr hohen Stromverbrauch, da ihr Sicherheitskonzept auf den Kosten für viel Rechenleistung beruht. Alternative Konzepte, wie zum Beispiel Proof-of-Stake, sind deutlich energieeffizienter und man kann bei ihnen ein stetiges Marktwachstum beobachten. Manche Netzwerke wie zum Beispiel Etherenum, wechseln aufgrund des hohem Energieverbrauches und der damit verbundenen gesellschaftlichen Kritik bereits zum Proof-of-Stake-Konzept. Es ist also nicht abwegig, dass der negative Umwelteinfluss von Kryptowährungen in Zukunft abnehmen wird.