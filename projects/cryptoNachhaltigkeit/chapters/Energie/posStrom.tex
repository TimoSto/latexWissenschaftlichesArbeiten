Im Vergleich zu den PoW-Tokens ist der Stromverbrauch von PoS-Tokens aufgrund des reduzierten Rechenaufwandes und der wegfallenden Redundanz vernachlässigbar (vgl. Kapitel 1.x). Dadurch könnte ein höherer Marktanteil erhofft werden, da das Thema Nachhaltigkeit in der Gesellschaft immer mehr an Bedeutung gewinnt\citebib{joule}{S.1843}{vgl. } Aus diesem Grund steigen einige Krypto-Netzwerke auf PoS um, wie zum Beispiel Ethereum es aktuell tut. Laut deren Einschätzungen wird der Energieverbrauch des PoS-Ethereum-Netzwerkes lediglich im MWh- oder niedrigen GWh-Bereich liegen\citebib{ethere}{}{vgl. }. Diese globalen Energieverbräuche liegen zum Beispiel noch deutlich unter dem Energieverbrauch der Stadt Münster im Jahr 2015 und sind deshalb vernachlässigbar\citebib{muenster}{}{vgl. }.