PoW-Tokens zeichnen sich durch den hohen Rechenaufwand zur Erstellung eines neuen Blocks aus. Dies liegt nicht nur an der aufwändigen Berechnung einer passenden Nonce, sondern auch an der Redundanz im Netzwerk. Viele Teilnehmer arbeiten gleichzeitig an der Hash-Berechnung und Erstellung eines Blocks, aber nur einer wird den Block an die Blockchain anhängen dürfen\citebib{sedlmeir}{S.604f.}{vgl. }.\\
Durch die unbestimmte Anzahl an aktiven Minern und deren Hardware wird die Bestimmung des tatsächlichen Stromverbrauchs eines Krypto-Netzwerkes unmöglich. Allerdings gibt es Ansätze, die minimal und maximal verbrauchte Energie anhand von technischen und ökonomischen Überlegungen abzuschätzen. Da der Rechenaufwand für die Validierung eines Blockes im Vergleich zu der Findung der Nonce vernachlässigbar ist, wird er in diesen Überlegungen nicht berücksichtigt\citebib{sedlmeir}{S.601}{vgl. }.