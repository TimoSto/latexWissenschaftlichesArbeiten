Aus den von Johannes Sedlmeir und seinen Kollegen ermittelten Zahlen lässt sich schließen, dass die Verbreitung und der Preis eines Coins in Korellation zur verbrauchten Energie stehen\citebib{sedlmeir}{S.601f.}{vgl. }. Dieser Zusammenhang entspricht dem Grundprinzip des \textit{homo oeconomicus}. Bei der Untergrenze des Verbrauchs spiegelt dies sich in der Anzahl der pro Stunde gefundenen Blöcke wider, da die Teilnehmer im Netzwerk bei rentableren Coins mehr Rechenaufwand investieren und folglich mehr Blöcke gefunden werden. Bei der Obergrenze ist die Entlohnung des Miners direkt vom Preis der Coins abhängig.\\
Wenn die Stromkosten für die Erstellung eines neuen Blockes also höher als die Belohnung sind, ist zu erwarten, dass Miner das Netzwerk verlassen\citebib{beincrypto}{}{vgl. }. Ein weiterer Grund könnte die mit dem Energieverbrauch verbundene CO\textsuperscript{2}-Belastung sein, da Nachhaltigkeit und Umweltschutz in der Öffentlichkeit immer mehr Aufmerksamkeit finden. Es gibt allerdings auch Argumentationen, dass der Energieverbrauch irrelevant sei, solange die Energie aus erneuerbaren Quellen stammt\citebib{joule}{S.1844}{vgl. }.