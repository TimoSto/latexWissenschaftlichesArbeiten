Die Berechnung des minimalen Stromverbrauchs eines Krypto-Netzwerkes orientiert sich an der sogenannten Schwierigkeit \textit{D} der Block-Erstellung. Diese spiegelt wider, wie viele Hashes \textit{H} berechnet werden müssen, bis eine passende Nonce gefunden wurde. Multipliziert man diesen Wert mit der Anzahl der stündlich entschlüsselten Blöcke \textit{n}, so erhält man die minimal benötigte Anzahl an Hash-Berechnungen \textit{R}, welche pro Stunde ausgeführt werden müssen.\citebib{sedlmeir}{S.601}{vgl. }
\begin{align}
    R_{min}\left[\frac{H}{s}\right]&=D\left[\frac{H}{B}\right]\cdot n\left[\frac{B}{s}\right]
\end{align}
\clearpage
\noindent Unter der optimistischen Annahme, dass alle aktiven Miner im Netzwerk die effizienteste Hardware benutzen, kann man die minimale Leistung \textit{P} eines Krypronetzwerkes ermitteln\citebib{sedlmeir}{S.601}{vgl. }:
\begin{align*}
    P_{min} [W]&\geq R_{min}\left[\frac{H}{s}\right]\cdot \frac{P\left[\frac{J}{s}\right]}{P_H\left[\frac{H}{s}\right]}\\
    &\geq R_{min}\left[\frac{H}{s}\right]\cdot E\left[\frac{J}{H}\right]
\end{align*}
Um auf den minimalen Stromverbrauch in einem Jahr zu kommen muss diese Leistung auf ein Jahr hochgerechnet werden:
\[ E_{min} [kWh] \geq P_{min}\cdot 24\cdot 365,25\]